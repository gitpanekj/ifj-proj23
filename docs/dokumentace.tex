\documentclass[11spt]{article}
\usepackage[left=2cm,top=2cm,text={17cm,24cm}]{geometry}
\usepackage[czech]{babel}
\usepackage[T1]{fontenc}
\usepackage[utf8]{inputenc}


\usepackage{picture}
\usepackage{graphics}
\usepackage{subfig}
\usepackage{pdflscape}
\usepackage{color}
\usepackage{enumerate}
\usepackage{array}
\usepackage[hidelinks]{hyperref}




\newcommand{\red}[1]{{\color{red} #1}}

\renewcommand{\labelenumii}{\theenumii}
\renewcommand{\theenumii}{\theenumi.\arabic{enumii}.}
\renewcommand{\theenumii}{\theenumi.\arabic{enumii}.}

\begin{document}

\begin{titlepage}
    \begin{center}
        \textsc{
        {\Huge Vysoké učení technické v~Brně}\\[0.5em]
        {\huge Fakulta Informačních technologií}} \\
        \vspace{\stretch{0.1}}
        {\Huge \textbf{Dokumentace překladače}} \\[0.4em]
        {\Large Projekt do předmětů: IFJ, IAL} \\[2em]
        {\Large Tým \textbf{xkotou08, Varianta\,--\,vv-BVS}}
    \end{center}
    \vspace{\stretch{0.6}}
    {\Large \textbf{Řešitelé}} \\[0.4em]
        {\Large
        \hspace*{0.5cm} (25\%) Lukáš Kotoun (xkotou08)\,--\,\textbf{vedoucí týmu}\\
        \hspace*{0.5cm} (25\%) Petr Novák (xnovak3l) \\
        \hspace*{0.5cm} (25\%) Jan Pánek (xpanek11) \\
        \hspace*{0.5cm} (25\%) Tibor Šimlaštík (xsimla00) \\
        }
    \vspace{\stretch{0.1}} \\
    {\Large \textbf{Rozšíření}} \\[0.4em]


    \vspace{\stretch{0.2}}
    {\Large \today}
\end{titlepage}


\tableofcontents
\newpage


\section{Úvod}
Dokumentace obsahuje popis jednotlivých částí překladače imperativního jazyka IFJ2023, který tvoří
podmnožinu jazyka Swift. V jednotlivých kapitolách jsou podrobněji popsány jednotlivé podčásti
překladače a formálná modely, na kterých jsou založeny.



\section{Lexikální analýza}
Logika lexikálního analyzátoru je implementována v~souboru \texttt{lexilcal\_analyzer.c}. 
Strukturu pro reprezentaci lexikálního analyzátoru \texttt{Scanner} obsahuje soubor \texttt{lexilcal\_analyzer.h}. \\
Strukturu pro reprzetanci tokenů společně s~definovanými typy tokenů a~operace nad tokeny obsahují soubory \texttt{tokens.h}, \texttt{tokens.c}.

Lexikální analyzátor je implementován na základě navrženého deterministického konečného automatu \href{fig:fsm}{\ref{fig:fsm}}, kdy jednotlivé podčásti automatu
jsou implementovány pomocí samostatných funkcí. Funkce realizující části automatu jsou volány z~hlavní funkce \texttt{scan\_token},
která sekvenci znaků načtenou ze standardního vstupu reprezentuje pomoci sturktury \texttt{Token}, kterou vrací. Pro propagaci chyby byly vyhrazeny
typy tokenů\\ \texttt{TOKEN\_LA\_ERROR} a~\texttt{TOKEN\_MEMMORY\_ERROR}.

Pro umoznění nahlednutí vpřed \texttt{Scanner} obsahuje buffer 3 následujích nezpracovaných znaků.
Náhlédnutí vpřed realizují funkce \texttt{peek}, \texttt{peek\_next} a \texttt{forward\_lookup}, která navíc umožňuje
porovnat obsah bufferu se specifikovaným formátem. Velikost bufferu 3 je dostatečná pro určení části automatu,
do které má být přechod uskutečněn bez nutnosti vracení znaků. Na základě obsahu bufferu je z~funkce \texttt{scan\_token} volána 
funkce implementující patřičnou část automatu.


Bíle znaky jsou zpracovány přimo ve funkci \texttt{scan\_token}. Zpracování jednořádkových komentářů implementuje funkce
\texttt{consume\_single\_line\_comment} a víceřádkových komentářů funkce \\ \texttt{consume\_multi\_line\_comment}. Jelikož vnořené
víceřádkové komentáře nelze modelovat pomocí samotného konečného automatu, bylo použito počítadlo vnoření.

Operátory, závorky a \texttt{\_}, \texttt{:}, \texttt{,}, \texttt{->} jsou zpracovány funkcí \texttt{scan\_operator}.
Číslicové literály jsou zpracovány ve funkci \texttt{scan\_number\_literal}.

Čás automatu pro identifikátory a~klíčová slova je implementována funkcí \texttt{scan\_identifier}.
Funkce \texttt{is\_kw} následně určí, zda se jméno načteného identifikátoru shoduje se jménem klíčového slova a~případně
vrací token pro dané klíčové slovo. Jinak vrací token pro daný identifkátor. U klíčových slov \texttt{Double}, \texttt{String}, \texttt{Int}
je ještě provedena kontrola, zda není následováno znakem \texttt{?}.

Částem automatu pro jednořádkové a~víceřádkové řetězce korespondují funkce \\ \texttt{scan\_single\_line\_string}, \texttt{scan\_multi\_line\_string}.
Řetězce jsou zpracovány dvěma průchody. Při prvním průchodu jsou načteny do bufferu a~je určeno, zda jsou korektně zakončeny. V druhém průchodu
jsou zpracovány obsažené escape sekvence a~u víceřídkových řetězcců ještě odsazení.

Pro účely syntaktické analýzy je u každého lexému, který nasleduje \texttt{$\backslash$n} nebo \texttt{$\{$}, nastaven příznak\\ \texttt{follows\_separator}.







\section{Tabulka symbolů}
Podle vybrané verze řešení projektu je námi implementována tabulka založena na výškové balancovaném stromu,
konkrétně samo-vyvažovací strom typu AVL. Každý element stromu ukládá informaci o svojí výšce, ukazatel na své dva potomky a~informace
o~symbolu které jsou o něm uložené v typu \texttt{symData}. Typ \texttt{symData} ukládá jméno symbolu, typ symbolu (proměnná nebo funkce),
datový typ proměnný nebo návratový typ funkce, zda je daný symbol konstanta, deklarovaný a~jestli byl inicializován.
Jestli symbol reprezentuje funkci, jsou v \texttt{symData} uložený též informace o~jeho parametrech (jejich počet a datové typy).

Ze vnějšku se pracuje s~typem symtable a~wrapper funkcemi začínající s~prefixem \texttt{symtable-}.
Typ symtable obsahuje kořen stromu typu \texttt{symtTreeElementPtr}. S tímhle typem a~se 
stromem samotným pracují rekurzivní funkce s~prefixem \texttt{symtTree-}.
Propagace změny struktury, tj. přidání/odstranění elementu ze stromu se vykonává pomocí návratové
hodnoty rekurzivních funkcí kterou je ukazatel na pozměnění podstrom.
Před návratem ukazatele je naň volaná funkce \texttt{symtTreeRebalance}, která aktualizuje výšku a~v~případě
potřeby vykoná rotaci podstromu a~vrací finální kořeň daného podstromu.

\section{Syntaktická a sémantická analýza}
Syntaktická analýza řídí celý překlad, proto pracuje skoro ze všemi moduly celého překladače.
Využívá lexikální analýzu pro postupné získávání tokenů, volá precedenční analýzu pro zpracování
výrazů, používá hierarchickou tabulku symbolů pro ukládání informací o proměnných a~funkcích a~zajišťuje generování výsledného kódu.


\subsection{Zpracování příkazů}
Syntaktická analýza je implementována jednoprůchodově pomocí metody rekurzivního sestupu
a~řídí se pomocí LL gramatiky a LL tabulky. Každé pravidlo z LL gramatiky je implementováno
jako samostatná funkce s~prefixem \texttt{rule\_}. 
Mezi jednotlivými funkcemi nejsou předávány parametry veškerá sdílená data jsou uložena v globálních proměnných.

Sémantická analýza je poté vepsána přímo do jednotlivých syntaktických funkcí a~využívá
další pomocné funkce pro kontrolu mnoha pravidel jako například: kontrola redefinice u definice proměnné nebo funkce,
kontrola kompatibility datových typů, nebo kontrola parametrů u~volání funkce. 

Pro jednoduší kontrolu parametrů u volání a~definice funkce byla využita datová struktura \\ \texttt{param\_vector}
pro ukládání typů parametrů a~jejich názvů. Vzhledem k implementaci jednoprůchodové analýzy, byla pro kontrolu volání
funkce před její definicí implementována pomocná funkce na odvození datových typů parametrů.
Odvození probíhá na základě datových typů použitích u jednotlivých volání.
Při definici funkce je poté zkontrolována kompatibilita definovaných datových typů s~odvozenými datovými typy použitými při volání. 

V~případě, kdy je odhalena~syntaktická nebo sémantická chyba je volána funkce \texttt{errorHandle},
která ukončí provádění programu s chybovým kódem pomocí instrukce exit.
Instrukce byla použita z~důvodu příliš\\ komplikovaného propagování chybového kódu.

\subsection{Zpracování výrazů}
Precedenční analýza je implementována na základě \href{tab:prec}{tabulky \ref{tab:prec}}.
Její programová podoba se nachází v~souboru \texttt{precedence\_analysis.h}.
Logika pro zpracování výrazů je implementována v~souboru \\
\texttt{precedence\_analysis.c}.
Při zpracování výrazů se využívá datové struktury \texttt{ExpressionStack},\\ \texttt{ExpressionStackItem} definované v souboru
\texttt{expression\_stack.h}. Zásobníková struktura je implementována jako nafukovací pole.
Operace nad \texttt{ExpressionStack} a \texttt{ExpressionStackItem} jsou implementovány v souboru \\ \texttt{expression\_stack.c}.

Klíčovou funkcí pro zpracování výrazů je \texttt{parse\_expression}, která na základě precedenční tabulky a aktuálního tokenu určí, jaká operace
má být provedena nad \texttt{ExpressionStack} (přidání prvku, započetí podvýrazu s vyšší prioritou, pokus o redukci pravida), nebo zda má být generován error.


V rámci precedenční analýzy rozlišujeme několik typů prvku zásobníku. Typy \texttt{LITERAL}, \texttt{IDENTIFIER}, \texttt{TERMINAL} reprezentují terminály.
Prvek s typem \texttt{EXPRESSION} potom vzniká při aplikaci redukčních pravidel.
Zásobník udržuje přehled o nejvrchnější přidaném terminálu pomocí indexu do pole realizujícího zásobník\\ \texttt{top\_most\_expr\_start}
a zároveň každý terminál obsahuje příznak \texttt{start\_of\_expr}, který indukuje, že terminál nad ním je již součástí podvýrazu s vyšší prioritou.
Sekvence prvků na indexech vyšších než \texttt{top\_most\_expr\_start} potom tvoří vrchol zásobníku.

Při aplikaci redukčního pravidla je na základě vrcholu zásobníku odvozeno číslo pravidla, které by se mělo aplikovat, pomocí funkce \texttt{get\_rule\_number}.


Zpracování výrazů je realizováno pomocí 2 průchodů. Při prvním průchodu jsou tokeny načítání do vektoru a je řešena syntaxe výrazu.
Zároveň je při prvním průchodu kontrolováno, zda se ve výrazu nevyskytuje operand s typem \texttt{Double}, \texttt{Double?}. Pokud
ano, je odvozen typ výsledného výrazu \texttt{Double}.

V druhém průchodu jsou následně prováděny sémantické akce a generování kódu.
Ukazatele na funcke realizující redukční pravidla jsou udržované v poli na indexu korespondujícím
číslu pravidla. Jednotlivé funcke implementují sémantické akce a generování kódu.












\section{Generování kódu}
Generování kódu je řešeno funkcemi v~souboru \texttt{codegen.c},  \texttt{codegen.h},
které jsou volány v~syntaktické analýze při volání určitých pravidel.
Příkazy pro interpret jsou psány rovnou na standartní výstup.
Pouze když se generuje kód, který je součástí cyklu while, tak se příkazy ukládají
do dynamického stringu, který je vypsaný po ukončení cyklu.
Definování proměnných se vypíše před tento string, protože se nemůže definovat proměnná se stejným názvem vícekrát.

Na začátku vykonávání SA se vypíše na výstup string, obsahující vestavěné funkce a~globální proměnné potřebné pro správný chod programu.

Na začátku volání funkce se nejdříve vytvoří nový dočasný rámec, do kterého se uloží hodnoty parametrů pojmenovaných číslem,
o~kolikátý parametr se jedná, ve formátu „\_číslo“. Na začátku definování funkce se dočasný rámec uloží do zásobníku
lokálních rámců, funkce potom pracuje s~lokálními či globálními proměnnými.
Poté se přeuloží hodnoty parametrů do proměnných se správným názvem parametru, aby se s~nimi lépe pracovalo.
Na konci funkce se návratová hodnota uloží na datový rámec a aktuální lokální rámec se vyjme ze zásobníků rámců.

Každá proměnná vytvořená uživatelem má k~sobě připojené číslo scopu, aby se docílilo jedinečnosti názvu proměnných.
Swift umožňuje překrývání proměnných v~různých scopech.

Podobně se přidává číslo při vytváření labelů pro if a~while, kde while i~if mají své unikátní počítadlo.
Při jejich ukončování se využívá zásobníku, který umožňuje jejich nekonečné zanořování.


Dále se generování kódu využívá v~precedenční analýze, kde se veškeré výrazy počítají na datovém zásobníku pomocí zásobníkových operací.
V~některých případech, kdy neexistují zásobníkové operace, se využívá globálních proměnných, které jsou definovány na začátku programu.
\section{Pomocné datové struktury}

Kromě již zmíněných datový struktur, jejichž popis byl pro větší přehledlenost zařazen do
konkrétních kapitol byly dále použity následující pomocné datové struktury.

Pro účely ukládání řetězců literálů a~identifikátorů slouží pomocná datová struktura \texttt{LiteralVector}.
Tato struktura umožňuje bufferování znaků tvořícího se literálu pomocí
\texttt{LV\_add} a jeho následné uložení pomocí \texttt{LV\_submit}.
Implementaci obsahují soubory \texttt{literal\_vector.c}, \texttt{literal\_vector.h}.

Pro ukládaní tabulek symbolu pro jednotlivé scopy byl vytvořen zásobníkový typ \texttt{symStack},
který je \\ naimplementovány na základě jednosměrného lineárního seznamu. Kromě základních funkcí
zásobníku umožňuje symstack i přechod pomocí ukazatele activeTable a funkcí \texttt{ActiveToTop} a~\texttt{Next}.
Element symstacku ukládá ukazatel na dynamicky alokovaný symtable a jedinečný identifikátor daného scopu.
Tabulka se Scope ID 0 je tabulka Globálního scopu.
Implementaci obsahují soubory \texttt{symstack.c}, \texttt{symstack.h}.

Pro kontrolu parametrů funkcí byla v~sémantické analýze použita datová struktura \texttt{ParamVector}.
Pro bufferování tokenů při prvním průchodu v precedenční analýze byla použita struktura \texttt{TokenVector}.\\
Implementace jsou popořadě obsaženy v souborech
\texttt{token\_vector.c}, \texttt{token\_vector.h}, \texttt{param\_vector.c}, \\ \texttt{param\_vector.h}.




\section{Rozdělení práce a plnění projektu}
Všichni členové týmy projevili značné usilí při plnění tohoto projektu a
nevyvstaly žádné problémy. Po rozdělení práce si všichni nastudovali potřebnou teorii
a ve spolupráci s ostatními včas dokončili požadovanou část. Komunikace probíhala 
prostřednictvím discordu a osobních setkání, která se konala jednou týdně. Pro verzování
jsme použili git hostovaný na GitHubu. 

Rozdělení práce: \\[1em]
\textbf{Lukáš Kotoun} (xkotou08): Syntaktická a sémantická analýza, gramatika, testování, dokumentace \\
\textbf{Petr Novák} (xnovak3l): Generování kódu, sémantická analýza, testování, dokumentace \\
\textbf{Jan Pánek} (xpanek11): Lexikální analýza, precedenční analýza, sémantická analýza, dokumentace \\
\textbf{Tibor Šimlaštík} (xsimla00): Tabulka symbolů, generování kódu, sémantická analýza, dokumentace\\





\newgeometry{left=1cm,top=3.5cm,text={17cm,24cm}}
\begin{landscape}
    \thispagestyle{empty}
    \begin{figure}[!h]
        \section{Přílohy}
        \subsection{Diagram konečného automatu}
        \begin{center}
        \scalebox{0.4}{\includegraphics{fsm_final.eps}}
        \caption{Diagram deterministického konečného automatu}
        \label{fig:fsm}
        \end{center}
    \end{figure}
\end{landscape}
\restoregeometry

\subsection{Gramatika}
\begin{enumerate}
    \item $\langle$prog$\rangle$ $\rightarrow$ $\langle$prog-body$\rangle$ \red{EOF}
    \item $\langle$prog-body$\rangle$ $\rightarrow$ $\langle$func-list$\rangle$ $\langle$statement-list$\rangle$ $\langle$prog-body$\rangle$
    \item $\langle$prog-body$\rangle$ $\rightarrow$ \red{\red{$\varepsilon$}}
    \item $\langle$statement-list$\rangle$ $\rightarrow$ $\langle$statement$\rangle$ $\langle$statment-list$\rangle$
    \item $\langle$statement-list$\rangle$ $\rightarrow$ \red{$\varepsilon$}
    \item $\langle$statement-func-list$\rangle$ $\rightarrow$ $\langle$statement-func$\rangle$ $\langle$statment-func-list$\rangle$
    \item $\langle$statement-func-list$\rangle$ $\rightarrow$ \red{$\varepsilon$}
    \item $\langle$func-list$\rangle$ $\rightarrow$ $\langle$func-decl$\rangle$ $\langle$func-list$\rangle$
    \item $\langle$func-list$\rangle$ $\rightarrow$ \red{$\varepsilon$}
    \item  $\langle$func-decl$\rangle$ $\rightarrow$ \red{FUNC ID (} $\langle$param-first$\rangle$ \red{)} $\langle$return-type$\rangle$ $\langle$func-body$\rangle$
    \begin{enumerate}
        \item $\langle$param-first$\rangle$ $\rightarrow$ $\langle$param$\rangle$ $\langle$param-n$\rangle$
        \item $\langle$param-first$\rangle$ $\rightarrow$ \red{$\varepsilon$}
        \item $\langle$param-n$\rangle$ $\rightarrow$ \red{,} $\langle$param$\rangle$ $\langle$param-n$\rangle$
        \item $\langle$param-n$\rangle$ $\rightarrow$ \red{$\varepsilon$}
        \item $\langle$param$\rangle$ $\rightarrow$ $\langle$param-name$\rangle$ $\langle$param-res$\rangle$ 
        \begin{enumerate}
            \item $\langle$param-name$\rangle$ $\rightarrow$ \red{\_}
            \item $\langle$param-name$\rangle$ $\rightarrow$ \red{ID}
            \item $\langle$param-res$\rangle$ $\rightarrow$ \red{\_ :} $\langle$type$\rangle$
            \item $\langle$param-res$\rangle$ $\rightarrow$ \red{ID :} $\langle$type$\rangle$
        \end{enumerate}
        \item $\langle$return-type$\rangle$ $\rightarrow$ \red{->} $\langle$type$\rangle$
        \item $\langle$return-type$\rangle$ $\rightarrow$ \red{$\varepsilon$}
    \end{enumerate}
    \item $\langle$type$\rangle$ $\rightarrow$ \red{INT}
    \item $\langle$type$\rangle$ $\rightarrow$ \red{INT-NIL}
    \item $\langle$type$\rangle$ $\rightarrow$ \red{DOUBLE}
    \item $\langle$type$\rangle$ $\rightarrow$ \red{DOUBLE-NIL}
    \item $\langle$type$\rangle$ $\rightarrow$ \red{STRING}
    \item $\langle$type$\rangle$ $\rightarrow$ \red{STRING-NIL}
    \item $\langle$func-body$\rangle$ $\rightarrow$ \red{$\{$} $\langle$statement-func-list$\rangle$ \red{$\}$}
    \item $\langle$body$\rangle$ $\rightarrow$ \red{$\{$} $\langle$statement-list$\rangle$ \red{$\}$}
    \item $\langle$statement-func$\rangle$ $\rightarrow$ \red{RETURN} $\langle$return-value$\rangle$
    \begin{enumerate}
        \item $\langle$return-value$\rangle$ $\rightarrow$ $\langle$expr$\rangle$
        \item $\langle$return-value$\rangle$ $\rightarrow$ \red{$\varepsilon$}
    \end{enumerate}
    \item $\langle$statement-func$\rangle$ $\rightarrow$ \red{IF} $\langle$if-cond$\rangle$ $\langle$func-body$\rangle$ \red{ELSE} $\langle$func-body$\rangle$
    \begin{enumerate}
        \item $\langle$if-cond$\rangle$ $\rightarrow$ \red{LET ID}
        \item $\langle$if-cond$\rangle$ $\rightarrow$ $\langle$expr$\rangle$
    \end{enumerate}
    \item $\langle$statement-func$\rangle$ $\rightarrow$ \red{WHILE} $\langle$expr$\rangle$ $\langle$func-body$\rangle$
    \item $\langle$statement-func$\rangle$ $\rightarrow$ $\langle$id-decl$\rangle$ $\langle$decl-opt$\rangle$
    \begin{enumerate}
        \item $\langle$id-decl$\rangle$ $\rightarrow$ \red{VAR ID}
        \item $\langle$id-decl$\rangle$ $\rightarrow$ \red{LET ID}
        \item $\langle$decl-opt$\rangle$ $\rightarrow$ \red{:} $\langle$type$\rangle$ $\langle$assign$\rangle$
        \begin{enumerate}
            \item $\langle$assign$\rangle$ $\rightarrow$ \red{$\varepsilon$}
            \item $\langle$assign$\rangle$ $\rightarrow$ \red{=} $\langle$statement-value$\rangle$
        \end{enumerate}
        \item $\langle$decl-opt$\rangle$ $\rightarrow$ \red{=} $\langle$statement-value$\rangle$
    \end{enumerate}
    \item $\langle$statement-func$\rangle$ $\rightarrow$ \red{ID} $\langle$statement-action$\rangle$
    \item $\langle$statement$\rangle$ $\rightarrow$ \red{IF} $\langle$if-cond$\rangle$ $\langle$body$\rangle$ \red{ELSE} $\langle$body$\rangle$
    \item $\langle$statement$\rangle$ $\rightarrow$ \red{WHILE} $\langle$expr$\rangle$ $\langle$body$\rangle$
    \item $\langle$statement$\rangle$ $\rightarrow$ $\langle$id-decl$\rangle$ $\langle$decl-opt$\rangle$
    \item $\langle$statement$\rangle$ $\rightarrow$ \red{ID} $\langle$statement-action$\rangle$
    \item $\langle$statement-action$\rangle$ $\rightarrow$ = $\langle$statement-value$\rangle$
    \item $\langle$statement-action$\rangle$ $\rightarrow$ ( $\langle$first-arg$\rangle$ )
    \begin{enumerate}
        \item $\langle$first-arg$\rangle$ $\rightarrow$ $\langle$arg$\rangle$ $\langle$arg-n$\rangle$
          \begin{enumerate}
                \item $\langle$arg$\rangle$ $\rightarrow$ $\langle$literal$\rangle$
                \item $\langle$arg$\rangle$ $\rightarrow$ \red{ID} $\langle$arg-opt$\rangle$
                \begin{enumerate}
                    \item $\langle$arg-opt$\rangle$ $\rightarrow$ \red{:} $\langle$term$\rangle$
                    \item $\langle$arg-opt$\rangle$ $\rightarrow$ \red{$\varepsilon$}
                \end{enumerate}
          \end{enumerate}
    \item $\langle$first-arg$\rangle$ $\rightarrow$ \red{$\varepsilon$}
    \item $\langle$arg-n$\rangle$ $\rightarrow$ \red{,} $\langle$arg$\rangle$ $\langle$arg-n$\rangle$
    \item $\langle$arg-n$\rangle$ $\rightarrow$ \red{$\varepsilon$}
\end{enumerate}
\item $\langle$statement-value$\rangle$ $\rightarrow$ \red{ID} $\langle$arg-expr$\rangle$
\begin{enumerate}
    \item $\langle$arg-expr$\rangle$ $\rightarrow$ \red{(} $\langle$first-arg$\rangle$ \red{)}
    \item $\langle$arg-expr$\rangle$ $\rightarrow$ $\langle$expr$\rangle$
\end{enumerate}
\item $\langle$statement-value$\rangle$ $\rightarrow$ $\langle$expr$\rangle$
\item $\langle$term$\rangle$ $\rightarrow$ \red{ID}
\item $\langle$term$\rangle$ $\rightarrow$ $\langle$literal$\rangle$
\begin{enumerate}
    \item $\langle$literal$\rangle$ $\rightarrow$ \red{INT-LITERAL}
    \item $\langle$literal$\rangle$ $\rightarrow$ \red{DOUBLE-LITERAL}
    \item $\langle$literal$\rangle$ $\rightarrow$ \red{STRING-LITERAL}
    \item $\langle$literal$\rangle$ $\rightarrow$ \red{NIL-LITERAL}
\end{enumerate}
\end{enumerate}



\begin{landscape}
    \subsection{LL-tabulka}
    \thispagestyle{empty}
    \begin{figure}[ht]
        \begin{center}
           \scalebox{0.40}{\includegraphics{ll1.eps}}
        \end{center}
        \caption{LL tabulka}
    \end{figure}
\end{landscape}

\begin{landscape}
    \thispagestyle{empty}
    \begin{figure}[ht]
        \begin{center}
            \scalebox{0.43}{\includegraphics{ll2.eps}}
        \end{center}
        \caption{LL tabulka\,--\,pokračování}
    \end{figure}
\end{landscape}


\subsection{Precedenční tabulka}



\begin{table}[ht]
    \catcode`\-=12
    \begin{center}
    \begin{tabular}{| c!{\vrule width 3pt} c | c | c | c | c | c | c | c | c | c | c | c | c | c | c | c |} \hline
              & SEP  & TERM &   (  &   )  &   +  &   -  &   *  &   /  &  ==  &  !=  &  <=  &  >=  &   >  &   <  &   !  &   ?? \\ \noalign{\hrule height 3pt}
         SEP  & x & < & < & > & < & < & < & < & < & < & < & < & < & < & < & <  \\ \hline
         TERM & > & x & x & > & > & > & > & > & > & > & > & > & > & > & > & > \\ \hline
         (    & > & < & < & = & < & < & < & < & < & < & < & < & < & < & < & <  \\ \hline
         )    & > & x & x & > & > & > & > & > & > & > & > & > & > & > & > & > \\ \hline
         +    & > & < & < & > & > & > & < & < & > & > & > & > & > & > & < & > \\ \hline
         -    & > & < & < & > & > & > & < & < & > & > & > & > & > & > & < & > \\ \hline
         *    & > & < & < & > & > & > & > & > & > & > & > & > & > & > & < & > \\ \hline
         /    & > & < & < & > & > & > & > & > & > & > & > & > & > & > & < & > \\ \hline
         ==   & > & < & < & > & < & < & < & < & x & x & x & x & x & x & < & <  \\ \hline
         !=   & > & < & < & > & < & < & < & < & x & x & x & x & x & x & < & <  \\ \hline
         <=   & > & < & < & > & < & < & < & < & x & x & x & x & x & x & < & <  \\ \hline
         >=   & > & < & < & > & < & < & < & < & x & x & x & x & x & x & < & <  \\ \hline
         >    & > & < & < & > & < & < & < & < & x & x & x & x & x & x & < & <  \\ \hline
         <    & > & < & < & > & < & < & < & < & x & x & x & x & x & x & < & <  \\ \hline
         !    & > & x & x & > & > & > & > & > & > & > & > & > & > & > & x & > \\ \hline
         ??   & > & < & < & > & < & < & < & < & < & < & < & < & < & < & < & <  \\ \hline
    \end{tabular}
    \end{center}
    \caption{Precedenční tabulka}
    \label{tab:prec}
\end{table}








\end{document}
